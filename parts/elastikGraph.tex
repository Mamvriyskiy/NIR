\chapter{Метод гибкого сравнения на графах}

Метод гибкого сравнения на графах -- метод обработки изображений и распознавания образов, который 
используется для нахождения соответствия между двумя изображениями, учитывая возможные искажения, изменение масштаба и повороты~\cite{distances}.

Данный метод является одним из способов распознования лиц.
Лица представлены в виде графов со взвешенными
вершинами и ребрами\cite{wen}. 
Во время распознавания один из графов –- константный(эталонный), 
в то время как другой изменяется(деформируется) с 
целью наилучшей подгонки к первому. 

В данном методе графы могут представлять собой как 
прямоугольную решетку (рис. \ref{img:ant}.a), так и структуру, образованную антропометрическими
точками лица(рис. \ref{img:ant}.б).

\begin{figure}[h]
    \centering
    \includegraphics[height=0.15\textheight]{img/ex.jpg}
    \caption{Пример структуры графа для распознования лиц \\ 
    a)Регулярная решетка; \\ б)Граф на основе антропометрических точек лица.}
    \label{img:ant}
\end{figure}

Частотное содержимое --- характеристика, показывающая насколько быстро 
меняется яркость или цвет в различных частях изображения.

Фильтр Габора --- фильтр, который анализирует, присутствует ли какое-либо конкретное частотное содержимое 
в изображении в различных направлениях в области вокруг точки или области анализа. Данный фильтр представляет собой 
синусоидальную плоскую волну(рис. \ref{img:sinuso}).\cite{gb}
\begin{figure}[h]
    \centering
    \includegraphics[width=0.35\textheight]{img/sin.png}
    \caption{Пример синусоидальной плоской волны.}
    \label{img:sinuso}
\end{figure}

\vspace{\baselineskip}
\vspace{\baselineskip}
\vspace{\baselineskip}
\vspace{\baselineskip}
\vspace{\baselineskip}
\vspace{\baselineskip}
\vspace{\baselineskip}

Свертка -- операция вычисления нового значения интенсивности, степени яркости
светлых пикселей по сравнению с более темными тонами,
заданного пикселя, при котором учитываются значения окружающих его соседних пикселей.

В некоторой локальной области вершины графа вычисляют значения
путем свертки значений яркости пикселей с набором(рис. \ref{img:gabnab}) фильтров Габора(рис. \ref{img:gab}).

\begin{figure}[h]
    \centering
    \includegraphics[height=0.35\textheight]{img/img_b.jpg}
    \caption{Набор фильтров Габора.}
    \label{img:gabnab}
\end{figure}
\vspace{\baselineskip}
\vspace{\baselineskip}
\vspace{\baselineskip}
\vspace{\baselineskip}
\vspace{\baselineskip}

\begin{figure}[h]
    \centering
    \includegraphics[height=0.15\textheight]{img/gbr1.png}
    \includegraphics[height=0.15\textheight]{img/gbr2.png}
    \includegraphics[height=0.15\textheight]{img/gbr3.png}
    \caption{Пример свертки изображения лица с фильтрами Габора.}
    \label{img:gab}
\end{figure}

\textbf{Алгоритм распознования лица:}
\begin{enumerate}
    \item Происходит деформация графа путем смещения каждой из его вершин на некоторое расстояние
    в различных направлениях относительно его исходного местоположения(рис. \ref{img:demonstration}).
    \item Выбирается такая позиция, при которой разница между значениями в вершине деформируемого графа и 
    соответствующей ей вершине эталонного графа будет минимальной.
    \item Данная операция выполняется поочердено для всех вершин графа, пока не будет достигнуто наименьшее суммарное различие 
    между признаками этих графов.
    \item  Данная процедура деформации должна выполняться для всех эталонных лиц, заложенных в базу данных системы. 
    Результат распознавания -- эталон с наименьшим суммарным различием.
\end{enumerate}


\begin{figure}[h]
    \centering
    \includegraphics[height=0.15\textheight]{img/img_a.jpg}
    \caption{Пример деформации графа.}
    \label{img:demonstration}
\end{figure}
    
