\chapter*{ВВЕДЕНИЕ}
\addcontentsline{toc}{chapter}{ВВЕДЕНИЕ}

В настоящее время, с ростом интереса к развитию технологий идентификации 
личности, особенно на основе изображений лиц, алгоритмы распознавания лиц 
приобретают все большее значение. Одним из ключевых направлений в области 
биометрических методов стало автоматическое обнаружение и идентификация 
лиц на изображениях, обеспечивая эффективные решения для таких 
разнообразных областей, как охранные системы, криминалистическая 
экспертиза, верификация, телеконференции, а также в сфере фотографии для 
автоматической фокусировки на лице человека.

Технология распознавания лиц на изображениях предоставляет преимущество 
перед другими биометрическими методами, поскольку не требует физического 
контакта с устройством. С учетом стремительного развития цифровой техники, 
она является наиболее приемлемой для массового применения. Однако, 
несмотря на свою эффективность, алгоритмы распознавания лиц сталкиваются 
с вызовами, такими как зависимость качества результата от условий 
освещенности, ракурса и положения лица, что требует постоянного 
совершенствования и анализа современных методов в данной области.

Цель данной работы --- классификация методов распознавания лиц в 
видеопотоке.

В рамках выполнения работы необходимо решить следующие задачи:

\begin{itemize}[label=---]
  \item провести анализ предметной области;
  \item провести классификацию подходов к решению;
  \item провести анализ методов.
\end{itemize}

