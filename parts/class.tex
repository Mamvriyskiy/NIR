\chapter{Классифкация методов распознования лиц в видеопотоке}

Методы распознавания лиц в видеопотоке можно классифицировать по различным критериям, таким как подход к представлению данных, использование обучения с учителем или без учителя и другие. В таблице \ref{tab:table} представлена данная классификация.

\begin{table}[h]
    \centering
    \small
    \caption{\label{tab:table} Классификация методов распознования лиц в видеопотоке}
    \begin{tabular}{|p{2cm}|p{3cm}|p{5cm}|p{5cm}|}
        \hline
        \multicolumn{1}{|l|}{\textbf{Методы}} & \multicolumn{3}{c|}{\textbf{Классификация}} \\
        \cline{2-4}
        & \textbf{по типу обучения} & \textbf{по подходу к представлению данных} & \textbf{по применению в распознавании лиц} \\
        \hline
        Метод главных компонент & Без учителя (преобразование данных) & Использует проецирование изначальных данных на главные компоненты & Используется для снижения размерности признакового пространства при работе с изображениями лиц \\
        \hline
        Метод гибкого сравнения на графах & В зависимости от реализации, может быть как с учителем, так и без учителя & Использует графы для представления структурной информации о лицах & Используется при работе с изображениями, где важна структура и отношения между частями лица \\
        \hline
        Метод опорных векторов & С учителем (обучение с классифицированными примерами) & Использует векторы признаков для построения гиперплоскости, разделяющей классы & Используется в задачах классификации лиц, особенно в случаях, когда есть различные классы (например, различные люди) \\
        \hline
        Метод Виолы-Джонса & С учителем (обучение на положительных и отрицательных примерах) & Использует характеристики (признаки), основанные на интенсивности пикселей & Применяется для обнаружения лиц в реальном времени, основываясь на быстрых вычислениях признаков \\
        \hline
    \end{tabular}
\end{table}
