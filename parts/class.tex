\chapter{Классифкация методов распознования лиц в видеопотоке}

Методы распознавания лиц в видеопотоке можно классифицировать по различным критериям, таким как подход к представлению данных, использование обучения с учителем или без учителя и другие. В таблице \ref{tab:table} представлена данная классификация.

\begin{table}[h]
    \centering
    \small
    \caption{\label{tab:table} Классификация методов распознования лиц в видеопотоке}
    \begin{tabular}{|p{3cm}|p{4.5cm}|p{4cm}|p{4cm}|}
        \hline
        \multirow{4}{*}{\textbf{Методы}} & \multicolumn{3}{c|}{\textbf{Классификация}} \\
        \cline{2-4}
        & \textbf{по подходу к представлению данных} & \textbf{по типу обучения} & \textbf{по применению в распознавании лиц} \\
        \hline
        Метод главных компонент & Метод главных компонент используется для уменьшения размерности данных путем проецирования их на главные компоненты. & Без учителя (преобразование данных). & Используется для снижения размерности признакового пространства при работе с изображениями лиц. \\
        \hline
        Метод гибкого сравнения на графах & Использует графы для представления структурной информации о лицах. & В зависимости от реализации, может быть как с учителем, так и без учителя. & Используется при работе с изображениями, где важна структура и отношения между частями лица. \\
        \hline
        Метод опорных векторов & Использует векторы признаков для построения гиперплоскости, разделяющей классы. & С учителем (обучение с классифицированными примерами). & Используется в задачах классификации лиц, особенно в случаях, когда есть различные классы (например, различные люди). \\
        \hline
        Метод Виолы---Джонса & Использует характеристики (признаки), основанные на интенсивности пикселей. & С учителем (обучение на положительных и отрицательных примерах). & Применяется для обнаружения лиц в реальном времени, основываясь на быстрых вычислениях признаков. \\
        \hline
    \end{tabular}
\end{table}
