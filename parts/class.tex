\chapter{Классифкация методов распознования лиц}

Методы распознавания лиц в видеопотоке можно классифицировать по различным критериям, таким как подход к представлению данных, использование обучения с учителем или без учителя и другие. Ниже представлена классификация рассматриваемых методов:

\begin{enumerate}
    \item\textbf{Метод главных компонент:}
    \begin{itemize}
        \item \textbf{Подход к представлению данных:} Метод главных компонент используется для уменьшения размерности данных путем проецирования их на главные компоненты.
        \item \textbf{Тип обучения:} Без учителя (преобразование данных).
        \item \textbf{Применение в распознавании лиц:} Используется для снижения размерности признакового пространства при работе с изображениями лиц.
    \end{itemize}
    
    \item\textbf{Метод гибкого сравнения на графах:}
    \begin{itemize}
        \item \textbf{Подход к представлению данных:} Использует графы для представления структурной информации о лицах.
        \item \textbf{Тип обучения:} В зависимости от реализации, может быть как с учителем, так и без учителя.
        \item \textbf{Применение в распознавании лиц:} Используется при работе с изображениями, где важна структура и отношения между частями лица.
    \end{itemize}
    
    \item\textbf{Метод опорных векторов:}
    \begin{itemize}
        \item \textbf{Подход к представлению данных:} Использует векторы признаков для построения гиперплоскости, разделяющей классы.
        \item \textbf{Тип обучения:} С учителем (обучение с классифицированными примерами).
        \item \textbf{Применение в распознавании лиц:} Используется в задачах классификации лиц, особенно в случаях, когда есть различные классы (например, различные люди).
    \end{itemize}
    
    \item\textbf{Алгоритм Виолы-Джонса:}
    \begin{itemize}
        \item \textbf{Подход к представлению данных:} Использует характеристики (признаки), основанные на интенсивности пикселей.
        \item \textbf{Тип обучения:} С учителем (обучение на положительных и отрицательных примерах).
        \item \textbf{Применение в распознавании лиц:} Применяется для обнаружения лиц в реальном времени, основываясь на быстрых вычислениях признаков.
    \end{itemize}
    
\end{enumerate}